%%%%%%%%%%%%%%%%%%%%%%%%%%%%%%%%%%%%%%%%%
% Authors:
% Marion Lachaise & François Févotte
% CC BY-NC-SA 3.0 (http://creativecommons.org/licenses/by-nc-sa/3.0/)
%%%%%%%%%%%%%%%%%%%%%%%%%%%%%%%%%%%%%%%%%

\documentclass{article}
\input{setup.tex} % Include the file specifying the document structure and custom commands

\usepackage{hyperref}
\hypersetup{
	colorlinks = false,
	pdfborder = {0 0 0 [3 3]}
}

\usepackage[dutch]{babel}

\title{Zonnecellen} % Title of the assignment
% \author{Yukiko Amagi\\ \texttt{y.amagi@inabauniversity.jp}} % Author name and email address
\date{Practicum natuurkunde, KU Leuven} % University, school and/or department name(s) and a date

\begin{document}
\maketitle

\section*{Inleiding}
Zonnecellen zetten stralingsenergie van de zon om in elektrische energie.
In eerste instantie werden ze alleen gebruikt op plaatsen waar geen andere elektriciteitsvoorziening is, zoals bij lichtboeien op zee en bij satellieten die rond de aarde draaien.
Maar tegenwoordig zie je ook op bedrijfsterreinen en in woonwijken steeds meer panelen met zonnecellen op de daken.
In dit practicum onderzoeken jullie hoe de stroom opgewekt door een zonnecel afhangt van de afstand tot een lichtbron en hoe het vermogen geleverd door een zonnecel afhangt van de opgewekte stroom.
Als extra: hier is een
\href{https://www.youtube.com/watch?v=xKxrkht7CpY}{\color{blue}TED-filmpje over hoe zonnecellen werken}.

\section{Theorie}
Het fotovolta\"ische effect doet zich voor wanneer een foton wordt geabsorbeerd door een materiaal dat is samengesteld uit gedoteerde n-type (negatief) en p-type (positief) halfgeleiders, pn-overgang genoemd.
Als gevolg van de dotering is er in het materiaal een permanent elektrisch veld aanwezig.
Tijdens de wisselwerking van een invallend foton (lichtdeeltje) met de elektronen van het materiaal, draagt het foton zijn energie over op het elektron dat is vrijgekomen uit zijn valentieband en wordt het dus blootgesteld aan het intrinsieke elektrische veld.
Onder invloed van dit veld migreert het elektron naar de bovenzijde.
Het gat dat aldus ontstaat, evolueert in de tegenovergestelde richting.
Door elektroden te plaatsen op de boven- en onderzijde, kunnen de elektronen worden aangetrokken en kan deze spanningsbron worden aangewend om elektrische stroom op te wekken.
Dit process is ge\"illustreerd in Fig.~\ref{fig:model}(a).

\begin{figure}[!hb]
  \centering
  \includegraphics[width=0.7\textwidth]{figuren/models.png}
	\caption{Eenvoudig model van een zonnecel en een equivalent circuit. (a) zonnecel (b) model circuit voor een ideale zonnecel. Figuur overgenomen uit \cite{bana}.}
	\label{fig:model}
\end{figure}

Een ideale zonnecel wordt gekarakteriseerd door een opgewekte stroom $I_{ph}$, die kan afwijken van het ideaal door elektrische of optische verliezen.
Het meest eenvoudige model voor een ideale zonnecel bestaat uit een foto-eletrische stroombron en een diode~\cite{bellia} en is ge\"illustreerd in Fig.~\ref{fig:model}(b).
De stroomsterkte opgewekt door de belichte zonnecel
\begin{equation}
	\label{eq:stroom}
	I = I_{ph} - I_d\,.
\end{equation}
De stroom van de diode $I_d$ duid de diffusie- en recombinatiestromen in de de pn-junctie en wordt benaderd door de Shockley vergelijking~\cite{bana}
\begin{equation}
	\label{eq:shockley}
	I_d = I_0\left[\exp\left(\frac{eV}{\eta k_B T}\right) - 1\right]\,,
\end{equation}
met $I_0$ de saturatiestroom, $V$ de `bias' spanning, $q$ de elementaire lading van de electron ($1.602\times10^{-19}$C), $k_B$ de constante van Boltzmann ($1.381\times10^{-23}$J/K), $T$ de temperatuur van de cel (K) en $\eta$ de idealiteitsfactor.

\begin{figure}[!hb]
  \centering
  \includegraphics[width=0.5\textwidth]{figuren/iv.jpg}
	\caption{De karakteristieke IV-curve voor het eenvoudig model van een ideale zonnecel. Figuur overgenomen uit \cite{bana}.}
	\label{fig:iv}
\end{figure}

De karakteristieke IV-curve dat volgt uit vergelijkingen \eqref{eq:stroom} en \eqref{eq:shockley} stelt alle mogelijke toestanden voor van je apparaat bij bepaalde testomstandigheden (zie Fig.~\ref{fig:iv}).
Stel, we voegen een weerstand $R$ toe in onze kring.
Als $R$ heel groot is, wordt de spanning gemeten die de zonnecel levert bij een stroomsterkte die vrijwel nul is.
Dit is de `open circuit voltage' en wordt aangeduidt met $V_{OC}$.
Als $R$ heel klein is, wordt de stroomsterke groot maar zakt de spanning tot vrijwel nul.
We spreken dan over `short circuit current' $I_{SC}$.
In beide gevallen is het door de zonnecel geleverde vermogen vrijwel nul.
In het tussenliggende gebied bereikt het vermogen ergens een maximum $P_{max}$.
Het opmeten van alle andere waarden van de karakteristieke IV-curve kan dus gedaan worden met een regelbare weerstand.
Specifiek voor de opstelling in dit practicum heb je ook twee multimeters nodig om de stroom en de spanning te meten.

\section{Opdrachten}
\vspace{-0.5cm}
\begin{question}
	Bepaal hoe de stroomsterkte $I$ van de zonnecel afhangt van de afstand $r$ tot de lichtbron.
	De stroom als functie van de afstand $r$, als $r$ niet te klein is, benaderd worden door
	\begin{equation}
		\label{eq:invsq}
		I(r) = \frac{I_a}{1+\frac{r^2}{a^2}}\,,
	\end{equation}
	waar $I_a$ en $a$ fitparameters zijn.
	\begin{enumerate}
		\item Meet voldoende datapunten voor $I(r)$.
		\item Maak een fit volgens vergelijking \eqref{eq:invsq}. Denk hierbij aan de geschikte opstelling van een hypothesetest en het bespreken van $\chi^2_{red}$, de p-waarde en de eenheden en fouten op de fitparameters.
	\end{enumerate}
\end{question}
\vspace{-0.8cm}
\begin{warn}[Let op:]
	Gebruik de multimeter bij de zonnecellen niet als een ohmmeter omdat de meetstroom de zonnecellen kan beschadigen.
	Controleer altijd de stand van de draaiknop.
\end{warn}
\vspace{-0.8cm}
\begin{question}
	Meet nu de IV-curve voor verschillende lichtsterkte aan de hand van de techniek van de regelbare weerstand.
	\begin{enumerate}
		\item Maak hiervoor een goede keuze voor enkele afstanden van de lichtbron tot de zonnecel (2-3 is genoeg). Fit het model in \eqref{eq:stroom} en \eqref{eq:shockley} aan de data om $I_0$ en $\eta$ zo te bepalen (denk aan een goede waarde voor $T$ die je kan kiezen). Bespreek ook hier je bevindingen met de juiste statistiek.
		\item Leg kort en bondig uit hoe de gevonden IV-curves veranderen t.o.v. de lichtsterkte.
		\item Duid ook op je figuur (probeer dit allemaal in \'e\'en figuur samen met de fit te doen) de $V_{OC}$, $I_{SC}$ en $P_{max}$ aan. Bespreek heel kort de gevonden waarden (met eenheden en fout erop uiteraard) in je tekst.
	\end{enumerate}
\end{question}

\begin{thebibliography}{9}

\bibitem{bellia}
	H. Bellia, R. Yousef en M. Fatima,
	\textit{A detailed modeling of photovoltaic module using MATLAB},
	NRIAG Journal of Astronomy and Geophysics,
	(2014), 3, 53-61

\bibitem{bana}
	S. Bana en R. P. Saini,
	\textit{A mathematical modeling framework to evaluate the performance of single diode and double diode based SPV systems},
	Energy Reports,
	(2016), 2, 171-187.

\end{thebibliography}

\end{document}

