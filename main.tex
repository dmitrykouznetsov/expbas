%%%%%%%%%%%%%%%%%%%%%%%%%%%%%%%%%%%%%%%%%
% Authors:
% Marion Lachaise & François Févotte
% CC BY-NC-SA 3.0 (http://creativecommons.org/licenses/by-nc-sa/3.0/)
%%%%%%%%%%%%%%%%%%%%%%%%%%%%%%%%%%%%%%%%%

\documentclass{article}
\input{setup.tex} % Include the file specifying the document structure and custom commands

\title{Foto-elektrisch effect} % Title of the assignment
% \author{Yukiko Amagi\\ \texttt{y.amagi@inabauniversity.jp}} % Author name and email address
\date{Practicum natuurkunde, KU Leuven} % University, school and/or department name(s) and a date

\begin{document}
\maketitle

\section*{Inleiding}

Bij het foto-elektrisch effect wordt gemeten hoeveel elektronen er vrij komen en hoe snel deze zich verplaatsen wanneer een metaal wordt beschenen met licht.
In 1905 publiceerde Einstein een Nobelprijs winnende paper waarin hij aantoonde dat de energie van de ontsnapte elektronen enkel afhangt van de frequentie van het licht en niet van de intensiteit van het licht.
Het foto-elektrisch effect is essentieel voor de werking van een fotocel dat in dit practicum zal gebruikt worden om de waarde van de constante van Planck te bepalen.
Voor een precieze bepaling van de constante van Planck zijn X-ray experimenten meer geschikt, maar het foto-elektrisch effect experiment is van groot historisch belang.

\section{Theorie}

Het foto-elektrisch effect was eerst waargenomen in 1887 door Heinrich Hertz.
Hij merkte op dat vonken konden overspringen tussen elektroden en dat dit effect het sterkst was voor ultraviolet licht.
Vanaf het begin van de $20^e$ eeuw, werd ontdekt dat licht een deeltjeskarakter bezit. Max Planck beschreef black-body radiation (in $1901$) door een holte te beschrijven als een gedempte harmonische oscillator, waaruit bleek dat de totale energie van de oscillator bestond uit discrete energiepakketjes die afhankelijk zijn van de frequentie
\begin{equation}
	\label{eq:planck}
	E=nhf\,,
\end{equation}
waar $E$ de energie is, $n\in\mathbb{N}$, $h$ de constante van Planck en $f$ de frequentie van de oscillator.
Einstein werkte dit idee in $1905$ uit en paste het toe op het foto-elektricsh effect.
Zijn besluit was dat elektronen alleen kunnen worden vrijgemaakt wanneer de frequentie van het invallende licht groot genoeg is.
Zo wordt de kinetische energie van de vrije elektron:
\begin{equation}
	\label{eq:einstein}
	E_{max} = hf - W\,,
\end{equation}
waarbij $W$ de uittreearbeid is (of ook de werkfunctie) die nodig is om het elektron vrij te maken uit het metaal.
\begin{info}
	Denk eens na waarom de kinetische energie in vergelijking \eqref{eq:einstein} maximaal is.
	Welke processen zouden deze energie kunnen verminderen?
\end{info}
Het opmeten van het aantal vrije elektronen die gevormd worden door het foto-elektrisch effect kan nauwkeurig gedaan worden door gebruikt te maken van een fotocel -- een anode en een kathode bedekt met een metaal met kleine uittreedarbeid, zoals Sb-Cs, waaruit door invallend licht elektronen vrijgemaakt worden.
De werking van de fotocel is afgebeeld in Fig.~\ref{fig:diagram}.
\begin{figure}[!ht]
	\centering
	\includegraphics[width=0.6\textwidth]{figuren/energy_diagram.png}
	\caption{Energie diagram voor elektronen in een fotocel.}
	\label{fig:diagram}
\end{figure}
Voor een fotocel is de kinetische energie van de vrije elektron gelijk aan
\begin{equation}
	\label{eq:1}
	W_{kin} = hf - W_C\,,
\end{equation}
met $W_C$ de uittreearbeid van de kathode.
Aan de hand van de wet van behoud van energie is er een bepaalde rempotentiaal $U_0$ dat ervoor zorgt dat de vrije elektronen net niet de anode bereiken en waardoor de stroom over de fotocel gelijk is aan nul.
Deze situatie is ook geschetst in Fig.~\ref{fig:diagram}.

De kinetische energie van de elektronen wanneer $U_0$ wordt aangelegd is ook afhankelijk van de onbekende `contact potentiaal' tussen elektrochemische potentialen van de anode en de kathode $U_A - U_C$, zodat $W_{kin}$ kan geschreven worden als
\begin{equation}
	\label{eq:2}
	W_{kin} = e(U_0 + U_A - U_C)\,,
\end{equation}
met $e$ de elementaire lading. Vergelijken van formules \eqref{eq:1} en \eqref{eq:2} geeft een lineair verband:
\begin{equation}
	\label{eq:linear}
	U_0 = \frac{h}{e} f - U_A\,,
\end{equation}
waaruit de waarde van de constante van Planck kan berekent worden alsook de uittreearbeid van de anode.

\begin{info}
	De uittreearbeid van de kathode komt niet voor in de uiteindelijke formule voor de remspanning.
Dit komt omdat de elektronen van de Fermi-niveau van de kathode komen en om het oppervlak van de anode te bereiken hebben ze dus al genoeg energie.
\end{info}

\section{Meetopstelling}

De gebruikte meetopstelling wordt getoond in Fig.~\ref{fig:opstelling}.
\begin{figure}[!ht]
	\centering
	\includegraphics[width=0.6\textwidth]{figuren/opstelling.png}
	\includegraphics[width=0.65\textwidth]{figuren/circuit.png}
	\caption{Een foto van de gebruikte opstelling (boven) en een schema van de opstelling (onder) met de belangrijkste componenten aangeduid.}
	\label{fig:opstelling}
\end{figure}
De opstelling zou reeds juist opgesteld zijn voor het begin van dit practicum, maar het is best om de volgende intellingen van de toestellen na te gaan:
\begin{itemize}
	\item De amplifier moet staan in `low drift mode' wat voor betere versterking van een zwak signaal moet zorgen. De amplificatie staat op $10^4$ met tijdsconstante $0.3$.
	\item Check het nulpunt van de amplifier: zet de multimeter op nul met niets aangesloten op de amplifier.
	\item Zet de voedingsbron op $3\si{\volt}$ en $1\si{\ampere}$.
	\item Maak een van de interferentiefilters vast aan de opening van de fotocel en plaats de fotocel direct voor de halogeenlamp. Selecteer de ronde aperture van de lamp met de slider.
	\item Verander de fotocel bias voltage
\end{itemize}
Het veranderen van de `bias voltage' wordt nu gedaan a.h.v. het regelen van de reostaat (regelbare weerstand).
De spanning over de amplifier is proportioneel met de opgewekte fotostroom.
Bij $10\si{\kilo\ohm}$ weerstand met een amplificatiefactor $10^4$, komt $1\si{\volt}$ bij de output overeen met een stroom van $10\si{\nano\ampere}$.

\begin{warn}[Let op:]
	Zet nooit een spanning hoger dan $12\si{\volt}$ op de Halogeenlamp.
	Dit zou de lamp geheid kapot maken!
\end{warn}

\section{Opdrachten}
\vspace{-0.5cm}
\begin{question}
	Bepaal de remspanning $U_0$ experimenteel voor verschillende lichtfrequenties en intensiteiten.
	Het opstellen van karakteristieke plots voor de fotocel doe je als volgt:
	\begin{enumerate}[a)]
		\item Plot $I$ t.o.v. de `bias voltage' $U$ voor alle kleurenfilters. Neem altijd genoeg datapunten om een mooie plot te verkrijgen!
		\item Plot $I$ t.o.v. $U$ voor 3+ verschillende afstanden van de lamp en de fotocel. Doe dit voor \'e\'en vaste waarde van $f$.
		\item Bespreek beide plots en bepaal hieruit de remspanning $U_0$.
	\end{enumerate}
\end{question}
\vspace{-0.8cm}
\begin{question}
	Bereken de constante van Planck $h$ aan de hand van de $U_0$--$f$ plot en bepaal ook de uittreearbeid $W_A$ van de anode.
	De berekende waarde voor $h$ kan sterk verschillen van de gekende waarde ($\pm 20\%$).
\end{question}

\end{document}

